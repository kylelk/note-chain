\documentclass[12pt,a4paper]{article}
\usepackage{lipsum}
\usepackage{authblk}
\usepackage[top=2cm, bottom=2cm, left=2cm, right=2cm]{geometry}
\usepackage{tocloft}
\usepackage{color}
\usepackage{hyperref}
\usepackage{filecontents}
%\usepackage{embedfile}

%
\renewenvironment{abstract}{%
\hfill\begin{minipage}{0.95\textwidth}
\rule{\textwidth}{1pt}}
{\par\noindent\rule{\textwidth}{1pt}\end{minipage}}
%
\makeatletter
\renewcommand\@maketitle{%
    \hfill
    \begin{minipage}{0.95\textwidth}
    \vskip 2em
    \let\footnote\thanks 
    {\LARGE \@title \par }
    \vskip 1.5em
    {\large \@author \par}
    \end{minipage}
    \vskip 1em \par
}
\makeatother

\renewcommand*\contentsname{Table of contents}

\pdfinfo{
   /Author (Nicola Talbot)
   /Title  (Creating a PDF document using PDFLaTeX)
   /CreationDate (D:20040502195600)
   /Subject (PDFLaTeX)
   /Keywords (PDF;LaTeX)
   /Date \date{}
}

\hypersetup{
	colorlinks,
	linktoc=all,
	linkcolor=black
}

%\embedfile[desc=LaTeX source code]{\jobname.tex}
\definecolor{light-gray}{gray}{0.95}
\newcommand{\codetext}[1]{\colorbox{light-gray}{\texttt{#1}}}

\begin{document}
%
%title and author details
\title{\textbf{Note Chain Project}}
\author{Kyle Kersey}
\date{}

\clearpage
\maketitle
%
\begin{abstract}
A note taking tool for managing simple text based notes containing many version
control features for managing revisions such as branching, commits. The
software is written in the Ada programming language and has a modular design
allowing use of different storage systems.
\end{abstract}

\renewcommand\cftsecleader{\cftdotfill{\cftdotsep}}

\tableofcontents
\thispagestyle{empty}
\pagebreak
\setcounter{page}{1}


\section{Project structure}

\subsection{Introduction}
The main source files are in the \codetext{src} directory and the compiled
result is put in the \codetext{obj} directory along with the object files and
other intermediate compiled files. Unit tests are in the \codetext{tests}
directory.

\subsection{Source files}
\subsubsection{main.adb}
This is the main file which loads the other packages and decides how to
interact with the other packages based on the command line arguments supplied
by the user. At the bottom of the file is a series of if else statements to
call a procedure for responding to a command. the methods which interact with a
command will by convention start with the \codetext{Cmd\_} prefix. When it
first starts it will call the \codetext{Init} procedure in the client package
passing in the database object as a parameter. At the end of the declaration
section a variable named \codetext{Data\_DB} which is the implementation of
Key-Value database being used. When Done the database connection is cleaned up.

\subsubsection{client.ads}
This package implements the main operations used in the project and is called
by \codetext{main.adb} at the start of the file the data types used in the
project are declared then the functions and procedures to operate on those
types. To make the system more modular this package is independent of the
database system being used and and will work with any Key-Value database which
implements the interface \codetext{KV\_Container}. Each of the methods which
interact with data need to be passed the variable containing the database, the
access mode to this variable needs to be \codetext{in out} write even for
operations which only read data because the database object might need to be
modified such as caching frequently used data.

\subsubsection{config.ads}
This package contains constant variables for project configuration values such as the version number and paths of where the project data is stored, file paths are made cross platform for Windows or Unix.

\subsubsection{file\_operations.ads}
Reusable operations for files are declared in this package such as writing to a file or getting the SHA-256 hash of a file this package also contains an operation called \codetext{Execute\_System\_Cmd} runs a shell command.

\subsubsection{settings.ads}
This stores the settings created by the user, the values are stored as key-value pairs in a hash map data structure then serialized to JSON when storing or loading from a file. The config package contains a constant named \codetext{Settings\_JSON\_File} which defines where this file is located. When a setting key is requested which does not exist then a 
\codetext{No\_Key\_Error} is raised.

\subsubsection{object\_store.ads}
Data is stored as as objects, this package contains operations on these objects such as loading and reading them from the Key-Value database, objects are identified by their SHA-256 hash. the methods in this package accept the database as Key-Value database as the first parameter. when an object cannot be found then a \codetext{Object\_Not\_Found} exception is raised. The exact format of an object is described in a later section.

\subsubsection{string\_operations.ads}
Reusable operations on strings are declared in this package such as converting a time object to and from a text format, it also has functions for searching in a string and validating the format of a SHA-256 hash

\subsubsection{note\_interactive\_menu.ads}
this shows a visual interactive menu to select a note. it returns the SHA-256 hash of the selected note, it uses the VIM editor to implement this.

\subsubsection{data\_store/kv\_store.ads}
This package defines an interface for Key-Value databases, new Key-Value databases must implement this interface to be used with the system. The interface ensures that child packages will implement operations such as set, get, remove and check if items exist the data store. It also has a procedure to commit values if the database being used as transactions which need to be committed to permanent storage. The procedure \codetext{Setup} is called to load the database and \codetext{Cleanup} is called when the database is done being used. If a requested item cannot be found in the database then the exception \codetext{No\_Key\_Error} should be raised. The name of a Key-Value store implementation is returned by the function \codetext{Name}.

\subsection{Unit tests}
Unit tests are done with the Aunit testing framework \cite{Aunit},  a test is written for each major package to unsure that the methods operate properly. For testing the data is stored in memory using a hash map key-value store.

\subsection{Data types and structures}
The main three data types used for the data are: Commit, Note and Tree. Each of These are record types which inherit from the Object\_Record type. Interface types are used to ensure that each of these data records will implement operations for working with the database. Ada does not have multiple inheritance, but does it allow for implementing multiple interfaces  \cite{Ada-Interface-Types}. The Interface \codetext{Persistable} is inherits from the types \codetext{JSON\_Serializable} and \codetext{Storable\_Object} this ensures that each of the record types which inherit from this interface will implement operations to serialize the data to and from a JSON \cite{rfc7159} String. The Storable\_Object interface makes records implement a procedure for saving to the database and a function for fetching an item from the database by it's SHA-256 hash.  \\

\par
All data is stored in an immutable data structure called a Merkle tree, each leaf in the tree contains the hash of the children leafs. A list of tree head nodes is maintained in a JSON file each item references a different branch name. The head node points to a commit object. A commit object contains when it was created, it's entry tree object reference and the next commit reference. The commits are like a liked list in that each item points to the next. the tree object contains a list of entries in that tree. Each entry has the type and reference of the entry, such as a Note object or another child tree. All data records items are stored as a JSON object. Before an item is put into the database it is prefixed with the object type name in lower case then the length of the data and a unix new line character then the Data Objects's content begins. This format is taken from how the git version control system stores items.

\subsubsection{Branches}

This is an example of the JSON file which stores the branches:

\begin{verbatim}
{
  "branches": {
    "master": {
      "name": "master",
      "commit_ref": "SHA-256 hash"
    }
  },
  "head": "master"
}
\end{verbatim}



\section{Usage tutorial}

\begin{filecontents}{\jobname.bib}
@misc{Aunit,
  author = "AdaCore",
  title = {\textit{Ada unit testing framework}},
  url = "http://libre.adacore.com/tools/aunit/",
  year = "2016"
}

@techreport{Ada-Interface-Types,
	title = {\textit{The Implementation of Ada 2005 Interface Types in the GNAT Compiler}},
	author = "Javier Miranda and Edmond Schonberg and Gary Dismukes",
	year = 2005,
	institution = "AdaCore",
	url = "http://www.adacore.com/uploads/technical-papers/Ada-2005_Interface_Types.pdf"
}

@techreport{rfc7159,
	title = {\textit{The JavaScript Object Notation (JSON) Data Interchange Format}},
	author = "Douglas Crockford",
	HOWPUBLISHED = {Internet Requests for Comments},
	year = 2014,
	month = "March",
	PUBLISHER = "{RFC Editor}",
	TYPE = "{RFC}",
	NUMBER = 7159,
	INSTITUTION = "{RFC Editor}"
}
\end{filecontents}

\bibliographystyle{unsrt}
\bibliography{\jobname}

\end{document}
